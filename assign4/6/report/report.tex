\documentclass[11pt]{article}
\usepackage{geometry, amsmath}
 \geometry{
 a4paper,
 total={170mm,257mm},
 left=20mm,
 top=20mm,
 }
%Gummi|065|=)
\title{\textbf{CS663 \\ Assignment 4 - Q6}}
\author{Kalpesh Krishna\\ Pranav Sankhe \\ Mohit Madan}
\date{}
\begin{document}

\maketitle
\textbf{Q6 a)} We have that $A$ is a $(m, n)$ matrix. $P = A^T A$ is a $(m, m)$ matrix. Hence $y$ is a $(m, 1)$ vector. We have,
\begin{align*}
y^tPy &= y^t A^T A y\\
&= (Ay)^T (Ay) \\
&= ||Ay||_2^{2} \\
&\geq 0
\end{align*}
Similarly, for $Q = AA^T$,
\begin{align*}
z^tQz &= z^t A A^T z\\
&= (A^Tz)^T (A^Tz) \\
&= ||A^Tz||_2^{2} \\
&\geq 0
\end{align*}
Hence both $P$ and $Q$ are positive semi-definite. For an eigenvalue $\lambda$ and eigenvector $v$ (assume $||v||_2 = 1$),
\begin{align*}
\lambda v &= Pv\\
\lambda v^T v &= v^T P v\\
\lambda &= v^T P v \geq 0
\end{align*}
Hence all eigenvalues are non-negative.\\\\\\
\textbf{Q6 b)} We have,
\begin{align*}
\lambda u &= Pu\\
\lambda Au &= APu \\
\lambda (Au) &= AA^T(Au) \\
\lambda (Au) &= Q(Au) \\
\end{align*}
Hence $Au$ is an eigenvector for $Q$, with the same eigenvalue. Similarly,
\begin{align*}
\mu v &= Qv\\
\mu A^Tv &= A^TQv \\
\mu (A^Tv) &= A^TA(A^Tv) \\
\mu (A^Tv) &= P(A^Tv) \\
\end{align*}
Hence $A^Tv$ is an eigenvector for $P$ with same eigenvalue. $u$ is a $(n, 1)$ sized vector and $v$ is a $(m, 1)$ sized vector.\\\\\\
\textbf{Q6 c)} We have,
\begin{align*}
& &\mu v_i &= Qv_i~\text{(definition of eigenvector)}\\
&\Rightarrow &\mu v_i &= AA^Tv_i\\
& &Au_i &= \frac{1}{||A^Tv_i||}A(A^Tv_i)\\
&\Rightarrow & &= \frac{\mu}{||A^Tv_i||}v_i = \gamma_iv_i\\
\end{align*}
Clearly, $\gamma_i \geq 0$ since $\mu \geq 0$ and $||A^Tv_i|| \geq 0$.\\\\\\
\textbf{Q6 d)} From the previous analysis we have,
\begin{align*}
&& Au_i &=  \gamma_iv_i\\
&\Rightarrow & A[u_1 u_2 ... u_m] &=  [\gamma_1v_1 \gamma_2v_2 ... \gamma_mv_m]\\
&\Rightarrow & AV &=  U \Gamma \\
&\Rightarrow & A &=  U \Gamma V^{-1}\\
&\Rightarrow & A &=  U \Gamma V^{T}~\text{since $V$ is orthonormal}\\
\end{align*}
Hence we show the existence of SVD. In the case $m > n$, the last few $\gamma_i$ are zero. In the case $n > m$, the remaining $u_i$ can be formed using Gram Schmidt process.
\end{document}
