\documentclass[11pt]{article}
\usepackage{geometry, amsmath, bm, amssymb}
 \geometry{
 a4paper,
 total={170mm,257mm},
 left=20mm,
 top=20mm,
 }
\newcommand{\evector}{\boldsymbol{e}}
\newcommand{\evone}{\boldsymbol{e_1}}
%Gummi|065|=)
\title{\textbf{CS663 \\ Assignment 4 - Q5}}
\author{Kalpesh Krishna\\ Pranav Sankhe \\ Mohit Madan}
\date{}
\begin{document}

\maketitle
Our minimizing objective function is $J$ such that (ignoring the terms not having $\evector$),
\begin{align*}
J(\evector) &= -\evector^T S \evector
\end{align*}
We have two constraints,
\begin{align*}
\evector ^T \evector &= 1 \\
\evector ^T \evone &= 0
\end{align*}
Writing a Lagrange multiplier equation (which we try to maximize),
\begin{align*}
J(\evector) &= \evector^T S \evector - \lambda (\evector ^T \evector - 1) - \mu (\evector ^T \evone)
\end{align*}
Taking derivative with respect to $\evector$ and equating to zero,
\begin{align*}
0 &= 2 S \evector - 2\lambda \evector - \mu \evone\\
0 &= 2\evone^T S \evector - 2\lambda \evone^T \evector - \mu~\text{(Multiplying $\evone^T$ on both sides)}\\
0 &= 2 \evone^T S^T \evector - 0 - \mu~\text{(Since S is symmetric)}\\
0 &= 2\lambda_1 \evone^T \evector - 0 - \mu\\
\therefore\mu &= 0 \\
\therefore \lambda &= \evector^T S \evector\\
\end{align*}
Hence, once again we need $\lambda$ to be an eigenvalue and $\evector$ to be an eigenvector. Since we cannot choose the largest eigenvalue (due to $\evone^T \evector = 0$), we use the second largest eigenvalue. $\evector$ is the unit eigenvector for the second largest eigenvalue.
\end{document}
