\documentclass[11pt]{article}
\usepackage{geometry, amsmath, bm, amssymb}
 \geometry{
 a4paper,
 total={170mm,257mm},
 left=20mm,
 top=20mm,
 }
\newcommand{\evector}{\boldsymbol{e}}
\newcommand{\evone}{\boldsymbol{e_1}}
%Gummi|065|=)
\title{\textbf{CS663 \\ Assignment 5 - Q2}}
\author{Kalpesh Krishna\\ Pranav Sankhe \\ Mohit Madan}
\date{}
\begin{document}

\maketitle
\section{Solution - 1D}
For a 1D image, we can assume the gradient kernel to be $h = [-1~~0~~1]$ (zero-crossing kernel). Since we have $g = h \star f$, we have $G = HF$ after taking their Discrete Fourier Transform. Hence this would imply $f = F^{-1}(G / H)$. Let's try to compute the DFTs of these signals to understand the issue with this operation.\\\\
Let's assume the length of our image is $K$. Hence we would need to zero-pad our filter $K-1$ times before finding its DFT. Hence we would have to take an $N$-point DFT with $N=K-1+3=K+2$
\begin{align*}
H(k) &= \sum_{n} h[n]e^{-j\frac{2\pi}{N}kn}\\
&= -1 + e^{-j\frac{4\pi}{N}k}, k \in \{0, 1, 2~...~N-1\}\\
\end{align*}
Now for $k=0$, we would have $H(k)=0$. (In other words, a gradient operation removes all DC components from a signal). Also for large values of $K$, $H(k)$ will be close to $1$ (in the complex domain).\\
Hence the student will not be able to recover the DC components in the signal and might have a hard time uncovering the larger frequencies if the signal is very long and has non-trivial high frequency components.
\section{Solution - 2D}
Similar to the 1D case, we attempt to calculate the DFT of the 2-D kernel. We use the kernel $h_x = [-1~0~1;-2~0~2;-1~0~1]$ for derivatives along the X-axis and $h_y = [1~2~1;0~0~0;-1-2-1]$ for derivatives along the Y-axis. Assuming we need to take a $N_1,N_2$-DFT after zero padding we obtain,
\begin{align*}
H_x(k_1, k_2) &= \sum_{x} \sum_{y} h_x[x,y]e^{-j\frac{2\pi}{N_1}k_1x}e^{-j\frac{2\pi}{N_2}k_2y}\\
&= (-1 + e^{-j\frac{4\pi}{N_1}k_1})(1 + 2e^{-j\frac{2\pi}{N_2}k_2} + e^{-j\frac{4\pi}{N_2}k_2}), k_1 \in \{0, 1, 2~...~N_1-1\}, k_2 \in \{0, 1, 2~...~N_2-1\}\\
H_y(k_1, k_2) &= \sum_{x} \sum_{y} h_y[x,y]e^{-j\frac{2\pi}{N_1}k_1x}e^{-j\frac{2\pi}{N_2}k_2y}\\
&= (1 - e^{-j\frac{4\pi}{N_2}k_2})(1 + 2e^{-j\frac{2\pi}{N_1}k_1} + e^{-j\frac{4\pi}{N_1}k_1}), k_1 \in \{0, 1, 2~...~N_1-1\}, k_2 \in \{0, 1, 2~...~N_2-1\}\\
\end{align*}
Quite clearly, for the DC case ($k_1 = k_2 = 0$) we obtain $H_x = H_y = 0$ and we won't be able to recover the DC components correctly. Again, for large widths (for X-derivatives) and large heights (for Y-derivatives), for $k_1 \rightarrow N_1$ and $k_2 \rightarrow N_2$.
\end{document}
